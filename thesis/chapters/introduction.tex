\chapter{Introduction}
\label{ch:introduction}

\section{History}

The miniaturization and economical availability of technological systems has
lead to their massive adoption in agriculture. Armed with data from sensors and
external sources such as local weather, farmers who implement precision
agriculture are gaining deeper understanding of their operations. This enables
them to better manage key resources while increasing productivity. Furthermore
machines today are capable of analyzing and storing large amounts of data which
wasn't possible a few years ago. The benefits of precision agriculture can be
already seen in greater cost savings and higher yield. Greater precision in
farming has already proved its benefits. Focusing mostly on data collection,
analysis, and visualization decision-making for the farmers can be greatly
eased. This is now known as Smart Farming. Data-driven farming is on course to
reshape the entire agricultural sector.

\section{Motivation}

Animals are naturally dynamic and their behavior is hard to predict. Caged or
farm animals can be easily monitored within the confines of the facility. Their
behavior and movement can be controlled. However in the case of free-range
animals, where the animal is not confined and has access to the outdoors, this
is very difficult. In such a case the animal's behavior is hard to control and
predict.

The farmer needs a system which can help in the decision making of the
free-range farm. Several RFID systems are installed on a free-range pig farm in
Hungary to track the movement of pigs. The main purpose is to collect
information which data scientists can use to come up with models which can
provide useful information. 

The data scientists however need a robust system to deploy their models and put
them into production. This application fills the gap between the data scientist
and the farmer. It needs to be flexible as new data and new analysis can be
developed by the scientists. It should also be simple enough so that the farmer
can use it without much technical knowledge.

\section{Overview of the project}

I am developing a Django web application for data collection and analysis in
which a user can upload several text files of data. The data is then stored and
the analysis is automatically run in the background. The results are stored and
can be seen as visualizations. The gathered data can be browsed in the system.
The application is easily extendible and does not require a lot of development
knowledge. There are pre-defined places where any new feature can be plugged in
and become ready to use in production. This will allow not only rapid
prototyping but also it will provide continuous feedback to the farmer about the
work of the data scientists.

\section{Outline}

In Chapter \ref{ch:user}, we focus on how to use this web application from the
user's perspective. A user will upload new data files and the application will
run the analysis on newly uploaded data. It will then update the visualization
with the results

Chapter \ref{ch:developer} discusses the details of this program. The libraries
and the models used in this project are explained. The architecture of the web
application which makes it modular is explained in detail with an example.

The final chapter \ref{ch:conclusion} is the thesis conclusion which will sum
everything and we will discuss further improvements and best practices