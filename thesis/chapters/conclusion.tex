\chapter{Conclusion} % Conclusion
\label{ch:conclusion}

This thesis project provides a platform to the data scientist to put their
models into robust production and the user to get value from a simple UI
interface. The project uses familiar technology and the visualization library is
the one used already in the Data Science world. It aims to glue the user to the
analytics while at the same time making it far easier for the developer to
extend it. More features can be added to the application currently such as email
or sms notifications for alerts depending on the analytics. Due to the clear
separation of various elements the testing of the application becomes far easier
as each element is composed of many atomic and testable methods.

By taking advantage of version control and continuous integration, the
development is made faster as multiple separate features can be worked on in
different git branches while CI provides the developer with the confidence that
their changes do not break the existing functionality. Furthermore using GitHub
CI we can develop a continuous deployment strategy which will immediately ship
changes to the host server as they are developed to be available to the user.

Due to the generic nature of the application it can be used in various other
fields as well where real time systems are required. One such example can be
Trading where different strategies can be developed which run in real time in
the background while providing the trader with visual output. The speed of the
application is fast enough that it can react quick enough to changes in the
market. As such the application will be always in constant development.

The short term goal has been achieved by the project but for the farmer
further data models are required to provide total possible value. These models
are under development at ELTE and will be integrated in the project in the
future.

\section*{Acknowledgements}
First and foremost, I'd like to thank my supervisor, Dr. Tomás Horváth, for
assisting me in selecting a thesis topic and guiding me through the process of
thesis work.

Many thanks to Dr. Márta Alexy as she was a crucial source of information about
the the perspective of the farmer, as well as for the information about the pigs
at the farm. This was very helpful as the front-end was made very simple for the
user.

Special thanks to Rameshkumar Pai Rohit for his work on the pig data and without
which this project would not have an useable example.